\documentclass{article}

\usepackage{booktabs}
\usepackage{tabularx}

\title{Development Plan\\\progname}

\author{\authname}

\date{}

%% Comments

\usepackage{color}

\newif\ifcomments\commentstrue %displays comments
%\newif\ifcomments\commentsfalse %so that comments do not display

\ifcomments
\newcommand{\authornote}[3]{\textcolor{#1}{[#3 ---#2]}}
\newcommand{\todo}[1]{\textcolor{red}{[TODO: #1]}}
\else
\newcommand{\authornote}[3]{}
\newcommand{\todo}[1]{}
\fi

\newcommand{\wss}[1]{\authornote{blue}{SS}{#1}} 
\newcommand{\plt}[1]{\authornote{magenta}{TPLT}{#1}} %For explanation of the template
\newcommand{\an}[1]{\authornote{cyan}{Author}{#1}}

%% Common Parts

\newcommand{\progname}{ProgName} % PUT YOUR PROGRAM NAME HERE
\newcommand{\authname}{Team \#, Team Name
\\ Student 1 name
\\ Student 2 name
\\ Student 3 name
\\ Student 4 name} % AUTHOR NAMES                  

\usepackage{hyperref}
    \hypersetup{colorlinks=true, linkcolor=blue, citecolor=blue, filecolor=blue,
                urlcolor=blue, unicode=false}
    \urlstyle{same}
                                


\begin{document}

\begin{table}[hp]
\caption{Revision History} \label{TblRevisionHistory}
\begin{tabularx}{\textwidth}{llX}
\toprule
\textbf{Date} & \textbf{Developer(s)} & \textbf{Change}\\
\midrule
Sept 25 2022 & Edwin Do & Initial commit with outline\\
Sept 25 2022 & Edwin Do & Add team member roles\\
Sept 25 2022 & Timothy Chen & Added to 1 and 2\\
Sept 25 2022 & Edwin Do & Add workflow plan \\
... & ... & ...\\
\bottomrule
\end{tabularx}
\end{table}

\newpage

\maketitle

\indent The evolution of the material’s microstructure during thermal treatment can be followed by measuring changes to the electrical conductivity of the material. The electrical conductivity is measured using a four-point probe. A current source provides a small current that passes through the material and a nano-voltmeter measures the voltage. Using the current and voltage values, the resistance can be calculated. This can in turn be converted to conductivity if the dimensions of the sample are known. The goal of this project is to develop a Window’s based App which can be used to measure the conductivity. A current source and a nano- voltmeter will be provided. The App needs to connect to these devices and calculate the conductivity. The user should be able to control the acquisition rate as well as output the data (plots and text files). The equipment should operate under constant current and delta (current pulse) control.

\section{Team Meeting Plan}
Team meeting will be held weekly on Microsoft Teams, exact time will be determined based on everyone's availability.
The meeting will start as soon as all members are present. 
Goal of the meeting will be determined before the meeting beginnings. 
A Bi-weekly meeting will be set with the supervisor to ensure project is heading in the direction. 

\section{Team Communication Plan}
Commuication will take place on Microsoft Teams and Discord. 
Microsoft teams is the main source of commuication used to host meeting with the team and the supervisor. 
Discord commuication are for updates and meeting to resolve issues brought up. 
All team members are expected to respone to commucations regarding themselves within a day.

\section{Team Member Roles}
\indent Everyone will be responsible for any aspect of the project on a as-needed basis. Therefore, there will be no team leader.
This will require every team member to be flexible and be able to move from one domain to another (i.e. software to hardware, or hardware to software).
The roles listed below will outline each member's primary responsiblities.
\begin{table}[h!]
    \centering

	
    \begin{tabular}{p{0.3\textwidth} p{0.6\textwidth}}

    \toprule
    \textbf{Team Member} & \textbf{Role}\\

    \midrule{Edwin Do} & Software Developer. Primarily assisting the team with Git, project management and the development of the GUI. \\
    \midrule{Timothy Chen} & Software Developer. Primarily assisting the team with development of the GUI. \\
    \midrule{Abdul Nour Seddiki} & Hardware Engineer. Primarily assisting the team with reading the data from the nano-voltmeter and current source. \\
    \midrule{Tyler Magarelli} & Software Developer. Primarily assisting the team with development of the GUI. \\
    \midrule{Joseph Braun} & Hardware Engineer. Primarily assisting the team with reading the data from the nano-voltmeter and current source.  \\

    \bottomrule

    \end{tabular}
    \caption{Team Member Roles}
\end{table}


\section{Workflow Plan}

\noindent Our plan is to primarily use GitHub to manage our workflow. 
Milestones will be created on GitHub ahead of time to outline our high level goals and when each milestone needs to be completed by.
The deadline of the milestones would align with the schedule outlined in the course outline.
Each milestone will include all issues that would be resolved in order to complete the milestone. 
Issues will be classified using the labels on GitHub. The classifications that will be primarily used for the 
GitHub issues are 'bug', 'feature', 'help-wanted' and 'documentation'. 'feature' issues will be outlined prior to starting a 
new milestone and 'bug' and 'help-wanted' issues will be created on a as-needed basis. In addition, 'feature' issues will help distribute work to 
members on the team and 'bug' and 'help-wanted' issues are flexible for any member that is available to provide help. 
When a new issue is created, it must be assigned to a milestone and this can be the current milestone or a future milestone.
Documentation issues will be used for issues related to our document deliverables (LaTex). \\

\noindent Every member will always to be assigned to at least one issue to ensure progress and equal distribution of work.
A new branch will be created for each issue. Once an issue has been completed, 
a pull request will be created and all team members will review the request before merge. 
There is a template for each pull request and can be edited as needed, but should outline the work that is completed
and the problem/issue that it solves. \\

\noindent Additional issue categories can include 'software' and 'hardware' to inform the team which domain the issue falls under so 
that assistance can be provided promptly.

\section{Proof of Concept Demonstration Plan}

What is the main risk, or risks, for the success of your project?  What will you
demonstrate during your proof of concept demonstration to convince yourself that
you will be able to overcome this risk?

\section{Technology}

\begin{itemize}
\item Specific programming language
\item Specific linter tool (if appropriate)
\item Specific unit testing framework
\item Investigation of code coverage measuring tools
\item Specific plans for Continuous Integration (CI), or an explanation that CI
  is not being done
\item Specific performance measuring tools (like Valgrind), if
  appropriate
\item Libraries you will likely be using?
\item Tools you will likely be using?
\end{itemize}

\section{Coding Standard}

\section{Project Scheduling}

\wss{How will the project be scheduled?}

\end{document}