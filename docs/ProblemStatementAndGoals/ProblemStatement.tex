\documentclass[12pt, titlepage]{article}

\usepackage{xcolor}
\usepackage{fullpage}
\usepackage[round]{natbib}
\usepackage{multirow}
\usepackage{booktabs}
\usepackage{tabularx}
\usepackage{graphicx}
\usepackage{float}
\usepackage{soul}
\usepackage{hyperref}
\hypersetup{
    colorlinks=black,
    citecolor=black,
    filecolor=black,
    linkcolor=red,
    urlcolor=blue
}
\usepackage{ulem}
\usepackage[round]{natbib}

\usepackage{graphicx}

\title{Problem Statement and Goals\\\progname}

\author{\authname}

\date{}

%% Comments

\usepackage{color}

\newif\ifcomments\commentstrue %displays comments
%\newif\ifcomments\commentsfalse %so that comments do not display

\ifcomments
\newcommand{\authornote}[3]{\textcolor{#1}{[#3 ---#2]}}
\newcommand{\todo}[1]{\textcolor{red}{[TODO: #1]}}
\else
\newcommand{\authornote}[3]{}
\newcommand{\todo}[1]{}
\fi

\newcommand{\wss}[1]{\authornote{blue}{SS}{#1}} 
\newcommand{\plt}[1]{\authornote{magenta}{TPLT}{#1}} %For explanation of the template
\newcommand{\an}[1]{\authornote{cyan}{Author}{#1}}

%% Common Parts

\newcommand{\progname}{ProgName} % PUT YOUR PROGRAM NAME HERE
\newcommand{\authname}{Team \#, Team Name
\\ Student 1 name
\\ Student 2 name
\\ Student 3 name
\\ Student 4 name} % AUTHOR NAMES                  

\usepackage{hyperref}
    \hypersetup{colorlinks=true, linkcolor=blue, citecolor=blue, filecolor=blue,
                urlcolor=blue, unicode=false}
    \urlstyle{same}
                                


\begin{document}

\maketitle

\begin{table}[hp]
\caption{Revision History} \label{TblRevisionHistory}
\begin{tabularx}{\textwidth}{llX}
\toprule
\textbf{Date} & \textbf{Developer(s)} & \textbf{Change}\\
\midrule
Sept 25 & Edwin Do & Initial commit with names\\
Sept 25 & Edwin Do & Add list of goals and new table format\\
... & ... & ...\\
\bottomrule
\end{tabularx}
\end{table}

\section{Problem Statement}
The Department of Materials Science and Engineering would like to measure the microstructure changes
during thermal treatment by looking at the conductivity changes in real-time when a sample undergoes 
thermal treatment. The changes can occur very quickly and it is crucial to read real-time data at a high enough acquisition rate.
The equipment 
% Add more here
We need to create an application that is compatible with the window's based computer in the Department of Materials Engineering.
This computer has a
%PORT? 
port that is compatible with the existing nano-voltmeter. 
In addition, the application must be able to read and display real-time data at a high acquisition rate to provide the most accurate data.


\wss{You should check your problem statement with the
\href{https://github.com/smiths/capTemplate/blob/main/docs/Checklists/ProbState-Checklist.pdf}
{problem statement checklist}.}
\wss{You can change the section headings, as long as you include the required information.}

%To Do
\subsection{Problem}

%To Do
\subsection{Inputs and Outputs}

\wss{Characterize the problem in terms of ``high level'' inputs and outputs.  
Use abstraction so that you can avoid details.}

\subsection{Stakeholders}
The stakeholders of this project include Dr. Hatem Zurob (Supervisor) and 
anyone who is interested in observing the resistivity values in microstructures.

\subsection{Environment}
The environment of this project includes the use of a current source, nano-voltmeter,
a Window's based computer provided by the Department of Materials Engineering and Science.

%  5 goals
%  Real time monitoring of conductivity
%  Remote access of control of the application
%  Create a window's based app that can be installed easily
%  Control acquisition rate up to 100 times per second
%  Display data as plots and text files
\newpage
\section{Goals}
\begin{table}[h!]
    \centering
    \begin{tabular}{p{0.3\textwidth} p{0.6\textwidth}}
    
    \toprule
    \textbf{Goals} & \textbf{Reason and measurement}\\
   
    \midrule{Real-time monitoring of conductivity} & A key feature to measure conductivity changes during thermal treatment. This can be measured by comparing the acquisition rate and how quickly the data is updated in the GUI. \\
    \midrule{Remote access of the application} & There may be jobs that take an extensive amount of time to complete. This will allow the user to check on the progress remotely. This can be measured by testing how accurate the progress is updated on the remote device. \\
    \midrule{Window Based Application that can be easily installed} & The computer used to connect to the nano-voltmeter and the current source will be using Windows as its operating system. This can be measured by looking at how successful the installation is and the time it required to be installed. \\
    \midrule{Control acquisition rate up to 100 times per second} & An acquistion rate of 100 times per second is necessary to provide the required granularity so that the data will be useful. This can be measured by observing the acquisition rate of the equipment and its output. \\
    \midrule{Display data as plots and text files} & The data must be displayed as plots so provide a visual representation of the data. Outputting the data to text files can allow the data to be ready for other uses/applications. This can be measured by observing how successful and accurate the application outputs the results to plots and text files. \\
    
    \bottomrule
    
    \end{tabular}
    \caption{List of goals}
\end{table}


\newpage
\section{Stretch Goals} %Edit This
% 3 goals
% Achieve an acquistion rate of 150 times per second
% Cross platform compatibility
\begin{table}[h!]
    \centering
    \begin{tabular}{p{0.3\textwidth} p{0.6\textwidth}}
    
    \toprule
    \textbf{Stretch Goals} & \textbf{Reason and measurement}\\
   
    \midrule{Real-time monitoring of conductivity} & A key feature to measure conductivity changes during thermal treatment. This can be measured by comparing the acquisition rate and how quickly the data is updated in the GUI. \\
    \midrule{Remote access of the application} & There may be jobs that take an extensive amount of time to complete. This will allow the user to check on the progress remotely. This can be measured by testing how accurate the progress is updated on the remote device. \\
    \midrule{Window Based Application that can be easily installed} & The computer used to connect to the nano-voltmeter and the current source will be using Windows as its operating system. This can be measured by looking at how successful the installation is and the time it required to be installed. \\
    
    \bottomrule
    
    \end{tabular}
    \caption{List of stretch goals}
\end{table}

\end{document}