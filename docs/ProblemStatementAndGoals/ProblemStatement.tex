\documentclass{article}

\usepackage{tabularx}
\usepackage{booktabs}

\title{Problem Statement and Goals\\\progname}

\author{\authname}

\date{}

\input{../Comments}
%% Common Parts

\newcommand{\progname}{Measuring Microstructure Changes During Thermal Treatment} % PUT YOUR PROGRAM NAME HERE
\newcommand{\authname}{Team \#30, ReSprint
\\ Edwin Do
\\ Joseph Braun
\\ Timothy Chen
\\ Abdul Nour Seddiki
\\ Tyler Magarelli
} % AUTHOR NAMES                  

\usepackage{hyperref}
    \hypersetup{colorlinks=true, linkcolor=blue, citecolor=blue, filecolor=blue,
                urlcolor=blue, unicode=false}
    \urlstyle{same}
                                


\begin{document}

\maketitle

\begin{table}[hp]
\caption{Revision History} \label{TblRevisionHistory}
\begin{tabularx}{\textwidth}{llX}
\toprule
\textbf{Date} & \textbf{Developer(s)} & \textbf{Change}\\
\midrule
Sept 25 & Edwin Do & Initial commit with names\\
... & ... & ...\\
\bottomrule
\end{tabularx}
\end{table}

\section{Problem Statement}
The Department of Materials Science and Engineering would like to measure the microstructure changes
during thermal treatment by looking at the conductivity changes in real-time when a sample undergoes 
thermal treatment. The changes can occur very quickly and it is crucial to read real-time data at a high enough acquisition rate.
The equipment 
% Add more here
We need to create an application that is compatible with the window's based computer in the Department of Materials Engineering.
This computer has a
%PORT? 
port that is compatible with the existing nano-voltmeter. 
In addition, the application must be able to read and display real-time data at a high acquisition rate to provide the most accurate data.


\wss{You should check your problem statement with the
\href{https://github.com/smiths/capTemplate/blob/main/docs/Checklists/ProbState-Checklist.pdf}
{problem statement checklist}.}
\wss{You can change the section headings, as long as you include the required information.}

%To Do
\subsection{Problem}

%To Do
\subsection{Inputs and Outputs}

\wss{Characterize the problem in terms of ``high level'' inputs and outputs.  
Use abstraction so that you can avoid details.}

\subsection{Stakeholders}
The stakeholders of this project include Dr. Hatem Zurob (Supervisor) and 
anyone who is interested in observing the resistivity values in microstructures.

\subsection{Environment}
The environment of this project includes the use of a current source, nano-voltmeter,
a Window's based computer provided by the Department of Materials Engineering and Science.

% 5 goals
\section{Goals}
%  Real time monitoring of conductivity
%  Remote access of control of the application
%  Create a window's based app that can be installed easily
%  Control acquisition rate up to 100 times per second
%  Display data as plots and text files



\begin{tabular}{ |p{3cm}||p{3cm}|p{4cm}|}
    \hline
    Goal& Reason & How to measure\\
    \hline
   %  Copy the line below for each entry
   Description of goal & Reasoning & Measuring method \\
    \hline
   \end{tabular}

% 3 goals
\section{Stretch Goals}
% Achieve an acquistion rate of 150 times per second
% Cross platform compatibility

\begin{tabular}{ |p{3cm}||p{3cm}|p{4cm}|}
    \hline
    Goal& Reason & How to measure\\
    \hline
   %  Copy the line below for each entry
   Description of goal & Reasoning & Measuring method \\
    \hline
   \end{tabular}

\end{document}