\documentclass[12pt, titlepage]{article}

\usepackage{booktabs}
\usepackage{tabularx}
\usepackage{hyperref}
\hypersetup{
    colorlinks,
    citecolor=black,
    filecolor=black,
    linkcolor=red,
    urlcolor=blue
}
\usepackage[round]{natbib}
\usepackage{multirow}

%% Comments

\usepackage{color}

\newif\ifcomments\commentstrue %displays comments
%\newif\ifcomments\commentsfalse %so that comments do not display

\ifcomments
\newcommand{\authornote}[3]{\textcolor{#1}{[#3 ---#2]}}
\newcommand{\todo}[1]{\textcolor{red}{[TODO: #1]}}
\else
\newcommand{\authornote}[3]{}
\newcommand{\todo}[1]{}
\fi

\newcommand{\wss}[1]{\authornote{blue}{SS}{#1}} 
\newcommand{\plt}[1]{\authornote{magenta}{TPLT}{#1}} %For explanation of the template
\newcommand{\an}[1]{\authornote{cyan}{Author}{#1}}

%% Common Parts

\newcommand{\progname}{ProgName} % PUT YOUR PROGRAM NAME HERE
\newcommand{\authname}{Team \#, Team Name
\\ Student 1 name
\\ Student 2 name
\\ Student 3 name
\\ Student 4 name} % AUTHOR NAMES                  

\usepackage{hyperref}
    \hypersetup{colorlinks=true, linkcolor=blue, citecolor=blue, filecolor=blue,
                urlcolor=blue, unicode=false}
    \urlstyle{same}
                                


\begin{document}

\title{Verification and Validation Report: \progname} 
\author{\authname}
\date{\today}
	
\maketitle

\pagenumbering{roman}

\section{Revision History}

\begin{tabularx}{\textwidth}{p{2.5cm}p{3cm}X}
\toprule {\bf Date} & {\bf Developer} & {\bf Notes/Changes}\\
\midrule
Mar 8, 2023 & Timothy Chen & Added performance to the Nonfunctional Test Evaluation\\
Date 2 & 1.1 & Notes\\
\bottomrule
\end{tabularx}

~\newpage

\section{Symbols, Abbreviations and Acronyms}

\renewcommand{\arraystretch}{1.2}
\begin{tabular}{l l} 
  \toprule		
  \textbf{symbol} & \textbf{description}\\
  \midrule 
  T & Test\\
  MIN\_USER\_ACCEPT\_RATE & 90\% - minimum acceptance rate\\
  TARGET\_TIME & 60 seconds \\
  INTERACT\_TIME & 5 seconds \\
  MAX\_MISTAKE & 2 \\
  MAX\_CAPACITY & 8GB \\ 
  MIN\_UPTIME & 30 minutes \\ 
  MIN\_SAMPLE\_RATE & 60 samples per second\\
  TIME\_ACCEPTED & 1 second \\
  ACCEPTED\_SIGFIG & 3 decimals \\
  \bottomrule
\end{tabular}\\

\wss{symbols, abbreviations or acronyms -- you can reference the SRS tables if needed}

\newpage

\tableofcontents

\listoftables %if appropriate

\listoffigures %if appropriate

\newpage

\pagenumbering{arabic}

This document ...

\section{Functional Requirements Evaluation}

\section{Nonfunctional Requirements Evaluation}

\subsection{Usability}
\begin{tabular}{ |p{3cm}||p{3cm}|p{3cm}|p{3cm}|p{2cm}| }
  \hline
  \multicolumn{5}{|c|}{Usability Tests} \\
  \hline
  Requirement & Related Unit Tests & Description & Expected Result & Result\\
  \hline
  Afghanistan   & AF    &AFG&   004 & PASS\\
  Aland Islands&   AX  & ALA   &248 &\\
  Albania &AL & ALB&  008 &\\
  Algeria    &DZ & DZA&  012 &\\
  American Samoa&   AS  & ASM&016 &\\
  Andorra& AD  & AND   &020 &\\
  Angola& AO  & AGO&024 &\\
  \hline
 \end{tabular}
		
\subsection{Performance}
The following is the list of Non-functional Test performed on the application to evaluate the performance of the application in respect to the test requirement. Each test will be mapped to unit test that are related to the corresponding requirement.\\
\begin{tabular}{ |p{2.3cm}||p{2cm}|p{3cm}|p{4cm}|p{2cm}| }
  \hline
  \multicolumn{5}{|c|}{Performance Tests} \\
  \hline
  Test Requirement & Related Unit Tests & Description & Expected Result & Result\\
  \hline
  NF-PT1   & - & Checking the minimum sampling rate of the application.  & The sampling rate of the application will be equal or greater than \textsl{MIN\_SAMPLE\_RATE} & PASS\\
  \hline
  NF-PT2   & - & Checking the time required for parameters to reflect in the application.  & The parameters will reflect in the application by within \textsl{TIME\_ACCEPTED} & PASS\\
  \hline
  NF-PT3   & - & Checking the significant digits used for calculations and display in the application.  &  The significant digits seen and used in the application is accurate to \textsl{ACCEPTED\_SIGFIG}. & PASS\\
  \hline
  NF-PT4   & - & Checking the up-time of the application during and after usage.  &  The application will have a up-time equal to or more than \textsl{MIN\_UPTIME} after the user completes a set of tasks. & PASS\\
  \hline
 \end{tabular}

\subsection{etc.}
	
\section{Comparison to Existing Implementation}	

This section will not be appropriate for every project.

\section{Unit Testing}

\section{Changes Due to Testing}

\section{Automated Testing}
		
\section{Trace to Requirements}
		
\section{Trace to Modules}		

\section{Code Coverage Metrics}

\bibliographystyle{plainnat}
\bibliography{../../refs/References}

\newpage{}
\section*{Appendix --- Reflection}

The information in this section will be used to evaluate the team members on the
graduate attribute of Lifelong Learning.  Please answer the following questions:

\begin{enumerate}
  \item 
  \item 
\end{enumerate}

\end{document}