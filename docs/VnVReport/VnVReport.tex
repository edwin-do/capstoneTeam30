\documentclass[12pt, titlepage]{article}

\usepackage[shortlabels]{enumitem}
\usepackage{comment}
\usepackage{booktabs}
\usepackage{tabularx}
\usepackage{hyperref}
\usepackage{float}
\usepackage{soul}
\usepackage{changepage}
\usepackage{graphicx}
\hypersetup{
    colorlinks,
    citecolor=black,
    filecolor=black,
    linkcolor=red,
    urlcolor=blue
}
\usepackage[round]{natbib}

%% Comments

\usepackage{color}

\newif\ifcomments\commentstrue %displays comments
%\newif\ifcomments\commentsfalse %so that comments do not display

\ifcomments
\newcommand{\authornote}[3]{\textcolor{#1}{[#3 ---#2]}}
\newcommand{\todo}[1]{\textcolor{red}{[TODO: #1]}}
\else
\newcommand{\authornote}[3]{}
\newcommand{\todo}[1]{}
\fi

\newcommand{\wss}[1]{\authornote{blue}{SS}{#1}} 
\newcommand{\plt}[1]{\authornote{magenta}{TPLT}{#1}} %For explanation of the template
\newcommand{\an}[1]{\authornote{cyan}{Author}{#1}}

%% Common Parts

\newcommand{\progname}{ProgName} % PUT YOUR PROGRAM NAME HERE
\newcommand{\authname}{Team \#, Team Name
\\ Student 1 name
\\ Student 2 name
\\ Student 3 name
\\ Student 4 name} % AUTHOR NAMES                  

\usepackage{hyperref}
    \hypersetup{colorlinks=true, linkcolor=blue, citecolor=blue, filecolor=blue,
                urlcolor=blue, unicode=false}
    \urlstyle{same}
                                


\begin{document}

\title{Verification and Validation Report: \progname} 
\author{\authname}
\date{\today}
	
\maketitle

\pagenumbering{roman}

\section{Revision History}

\begin{table}[H]
  \caption{\bf Revision History}
  \begin{tabularx}{\textwidth}{p{2.5cm}p{2.5cm}X}
  \toprule {\bf Date} & {\bf Developer} & {\bf Notes/Changes}\\
  \midrule
  Mar 7, 2023 & Abdul Nour & Updated template\\
  Mar 8, 2023 & Abdul Nour & Added Functional Requirements Evaluation \& Trace to Requirements\\
  Mar 8, 2023 & Abdul Nour & Added Functional Req's Unit Tests\\
  \bottomrule
  \end{tabularx}
  \end{table}

~\newpage

\section{Symbols, Abbreviations and Acronyms}

\renewcommand{\arraystretch}{1.2}
\begin{tabular}{l l} 
  \toprule		
  \textbf{symbol} & \textbf{description}\\
  \midrule 
  T & Test\\
  \bottomrule
\end{tabular}\\

\wss{symbols, abbreviations or acronyms -- you can reference the SRS tables if needed}

\newpage

\tableofcontents

\listoftables %if appropriate

\listoffigures %if appropriate

\newpage

\pagenumbering{arabic}

%This document ...

\section{Functional Requirements Evaluation}

\noindent The following table suggests examples of system level tests that were executed in order to verify that the functional requirements of the system were met.


\begin{table}[H]
	\centering
	\caption{Test Cases for Functional Requirements}
	\label{my-label}
%	\begin{tabular}{|l|l|l|l|l|}
	\begin{tabular}{|p{0.85cm}|p{5cm}|p{5cm}|p{1.3cm}|}
		\hline
		\textbf{ID} & \textbf{User Action} & \textbf{Expected Result}  & \textbf{Result} \\ \hline
		ST1 & Check "Current Source" radio button, type (1) in the "Current Supply" textbox, type (10) in the "Compliance" textbox and click "Set" button, click "Current ON" button, check "Nano-Voltmeter" radio button, click "Start Capture" button & Current Source is reset, current  is set to 1mA and displayed, ompliance voltage is set to 10V, current is turned on, nanovoltmeter is reset and values from the voltmeter are continuously displayed on the application along with real-time calculations & PASS\\ \hline
		ST2 & System is capturing values, thermal treatment is taking place & Critical changes in resistivity are noted in the capture log & FAIL*\\ \hline
		ST3 & Choose an "Integration Rate" of (1 PLC) from the drop-down menu & Voltmeter display shows faster reading than default & PASS\\ \hline
		ST4 & System is capturing values, thermal treatment is taking place & Changes in resistivity-temperature slopes are noted in caputre log and displayed & FAIL*\\ \hline
		ST5 & Capture and current are on, click "Stop Experiment" button on remote interface & Capture and current supply are turned off & FAIL* \\ \hline
		ST6 & Start experiment and capture & Graph is displaying measurements and calculations & PASS \\ \hline
		ST7 & Enter experiment info \& conduct an experiment & File output contains all experiment data & PASS \\ \hline
	\end{tabular}
\end{table}
\noindent * This feature has not been implemented yet.

\newpage

\section{Nonfunctional Requirements Evaluation}

\subsection{Usability}
		
\subsection{Performance}

\subsection{etc.}
	
\section{Comparison to Existing Implementation}	

This section will not be appropriate for every project.

\section{Unit Testing}

\subsection{Functional Requirements Testing}

\begin{table}[H]
	\centering
	\caption{Unit Tests for Functional Requirements}
	\label{my-label}
	\begin{tabular}{|p{0.85cm}|p{5cm}|p{5cm}|p{1.3cm}|}
		\hline
		\textbf{ID} & \textbf{User Action} & \textbf{Expected Result}  & \textbf{Result}\\ \hline
		UT1 & Check "Current Source" radio button & Current Source is reset & PASS \\ \hline
		UT2 & Type an integer (1) in the "Current Supply" textbox and click "Set" button & Current Supply is set to 1mA and displayed & PASS \\ \hline
		UT3 & Type an integer (10) in the "Compliance" textbox and click "Set" button & Compliance voltage is set to 10V & PASS\\ \hline
		UT4 & Click "Current ON" button & Current Source is supplying current & PASS \\ \hline
		UT5 & Click "Current OFF" button & Current Source stops supplying current & PASS \\ \hline
		UT6 & Check "Nano-Voltmeter" radio button & Nano-Voltmeter is reset & PASS \\ \hline
		UT7 & Click "Start Capture" button & Values from the voltmeter are continuously displayed on the application along with real-time calculations & PASS  \\ \hline
		UT8 & Click "Stop Capture" button & Values and calcualtions stop generating, latest batch remains visible & PASS \\ \hline
		UT9 & Current supply is set up and capture is started & Accurate calculations of resistivity are displayed continuously & PASS \\ \hline
	\end{tabular}
\end{table}

%\newpage

\begin{table}[H]
	\centering
	\begin{tabular}{|p{1cm}|p{5cm}|p{5cm}|p{1.3cm}|}
		\hline
		\textbf{ID} & \textbf{User Action} & \textbf{Expected Result}  & \textbf{Result}\\ \hline
		UT10 & System is capturing values, thermal treatment is taking place & Critical changes in resistivity are noted in the capture log & FAIL*\\ \hline
		UT11 & System is capturing values, no thermal treatment & No critical changes in resistivity are noted & PASS\\ \hline
		UT12 & Choose an "Integration Rate" of (1 PLC) from the drop-down menu & Voltmeter display shows faster reading than default & PASS \\ \hline
		UT13 & Choose an "Integration Rate" of (5 PLC) from the drop-down menu & Voltmeter display shows no change than default rate & PASS\\ \hline
		UT14 & System is capturing values, thermal treatment is taking place & Changes in resistivity-temperature slopes are noted in caputre log and displayed & FAIL* \\ \hline
		UT15 & System is capturing values, no thermal treatment & No changes in resistivity-temperature slopes are noted in caputre log or displayed & PASS \\ \hline
		UT16 & Current supply is set up and capture is started, "Stop Experiment" button is clicked on remote interface & Capture and current supply are turned off & FAIL* \\ \hline
		UT17 & Current supply is set up and capture is started, remote interface is launched but not interacted with & Capture and current supply stay on (no change) & FAIL* \\ \hline
	\end{tabular}
\end{table}

\section{Changes Due to Testing}

\wss{This section should highlight how feedback from the users and from 
the supervisor (when one exists) shaped the final product.  In particular 
the feedback from the Rev 0 demo to the supervisor (or to potential users) 
should be highlighted.}

\section{Automated Testing}
		
\section{Trace to Requirements}

\begin{table}[H]
	\centering
	\caption{Requirements Traceability}
	\label{my-label}
	\begin{tabular}{|c|c|c|c|}
		\hline
		\textbf{System Test} & \textbf{Unit Tests} & \textbf{Requirement} & \textbf{Plan} \\ \hline
		ST1 & UT1 — UT9 & FR1 &FR-T1 \\ \hline
		ST2 & UT10, UT11 & FR2 &FR-T2\\ \hline
		ST3 & UT12, UT13 & FR3 &FR-T3\\ \hline
		ST4 & UT14, UT15 & FR4 &FR-T4\\ \hline
		ST5 & UT16, UT17 & FR5 &FR-T5\\ \hline
		ST6 &  & FR6 &FR-T6\\ \hline
		ST7 &  & FR7 &FR-T7\\ \hline
	\end{tabular}
\end{table}
		
\section{Trace to Modules}		

\section{Code Coverage Metrics}

\bibliographystyle{plainnat}
%\bibliography{../../refs/References}

\newpage{}
\section*{Appendix --- Reflection}

The information in this section will be used to evaluate the team members on the
graduate attribute of Reflection.  Please answer the following question:

\begin{enumerate}
  \item In what ways was the Verification and Validation (VnV) Plan different
  from the activities that were actually conducted for VnV?  If there were
  differences, what changes required the modification in the plan?  Why did
  these changes occur?  Would you be able to anticipate these changes in future
  projects?  If there weren't any differences, how was your team able to clearly
  predict a feasible amount of effort and the right tasks needed to build the
  evidence that demonstrates the required quality?  (It is expected that most
  teams will have had to deviate from their original VnV Plan.)
\end{enumerate}

\end{document}