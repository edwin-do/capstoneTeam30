\documentclass[12pt, titlepage]{article}

\usepackage{fullpage}
\usepackage[round]{natbib}
\usepackage{multirow}
\usepackage{booktabs}
\usepackage{tabularx}
\usepackage{graphicx}
\usepackage{float}
\usepackage{hyperref}
\usepackage{pdfpages}
\usepackage{pdflscape}

\hypersetup{
    colorlinks,
    citecolor=black,
    filecolor=black,
    linkcolor=red,
    urlcolor=blue
}
\usepackage[round]{natbib}
\usepackage{multirow}

%% Comments

\usepackage{color}

\newif\ifcomments\commentstrue %displays comments
%\newif\ifcomments\commentsfalse %so that comments do not display

\ifcomments
\newcommand{\authornote}[3]{\textcolor{#1}{[#3 ---#2]}}
\newcommand{\todo}[1]{\textcolor{red}{[TODO: #1]}}
\else
\newcommand{\authornote}[3]{}
\newcommand{\todo}[1]{}
\fi

\newcommand{\wss}[1]{\authornote{blue}{SS}{#1}} 
\newcommand{\plt}[1]{\authornote{magenta}{TPLT}{#1}} %For explanation of the template
\newcommand{\an}[1]{\authornote{cyan}{Author}{#1}}

%% Common Parts

\newcommand{\progname}{ProgName} % PUT YOUR PROGRAM NAME HERE
\newcommand{\authname}{Team \#, Team Name
\\ Student 1 name
\\ Student 2 name
\\ Student 3 name
\\ Student 4 name} % AUTHOR NAMES                  

\usepackage{hyperref}
    \hypersetup{colorlinks=true, linkcolor=blue, citecolor=blue, filecolor=blue,
                urlcolor=blue, unicode=false}
    \urlstyle{same}
                                


\begin{document}

\title{Verification and Validation Report: \progname} 
\author{\authname}
\date{\today}
	
\maketitle

\pagenumbering{roman}

\section{Revision History}

\begin{tabularx}{\textwidth}{p{3cm}p{3cm}X}
\toprule {\bf Date} & {\bf Developer} & {\bf Change}\\
\midrule
Date 1 & 1.0 & Notes\\
Mar. 8, 2023 & Joseph Braun & Added Section 5 \\
\bottomrule
\end{tabularx}

~\newpage

\section{Symbols, Abbreviations and Acronyms}

\renewcommand{\arraystretch}{1.2}
\begin{tabular}{l l} 
  \toprule		
  \textbf{symbol} & \textbf{description}\\
  \midrule 
  T & Test\\
  \bottomrule
\end{tabular}\\

\wss{symbols, abbreviations or acronyms -- you can reference the SRS tables if needed}

\newpage

\tableofcontents

\listoftables %if appropriate

\listoffigures %if appropriate

\newpage

\pagenumbering{arabic}

This document ...

\section{Functional Requirements Evaluation}

\section{Nonfunctional Requirements Evaluation}

\subsection{Usability}
\begin{tabular}{ |p{3cm}||p{3cm}|p{3cm}|p{3cm}|p{2cm}| }
  \hline
  \multicolumn{5}{|c|}{Usability Tests} \\
  \hline
  Requirement & Related Unit Tests & Description & Expected Result & Result\\
  \hline
  Afghanistan   & AF    &AFG&   004 & PASS\\
  Aland Islands&   AX  & ALA   &248 &\\
  Albania &AL & ALB&  008 &\\
  Algeria    &DZ & DZA&  012 &\\
  American Samoa&   AS  & ASM&016 &\\
  Andorra& AD  & AND   &020 &\\
  Angola& AO  & AGO&024 &\\
  \hline
 \end{tabular}
		
\subsection{Performance}

\subsection{etc.}
	

\section{Comparison to Existing Implementation}	

\noindent Below is an image of the exisiting implementation's GUI.

\begin{figure}[H]
\centerline{\includegraphics[scale=0.2]{app.png}}
\caption{Previous software user interface design}
\label{fig}
\end{figure}

\noindent The key elememts of the existing implementation will also be included in our implementation. This is so that the application will be familiar to the user and as intuitive as possible. These elements are listed below:
\begin{itemize}
  \item Method Panel: controls which device to read data from (tempertaure or voltage), and which file to save the data in
  \item Device Settings Panel: used to send SCPI commands directly to a device
  \item Readout: includes graphical output and listed output of relevant values (current, voltage, resistance, etc.)
\end{itemize}

\noindent The primary difference between the existing implementation and our implementaion is the appearance of the GUI. The exisiting implementation was developed for Windows XP and in its current state is non-functional. Our implementation is developed for Windows 10. 

\section{Unit Testing}

\section{Changes Due to Testing}

\section{Automated Testing}
		
\section{Trace to Requirements}
		
\section{Trace to Modules}		

\section{Code Coverage Metrics}

\bibliographystyle{plainnat}
\bibliography{../../refs/References}

\newpage{}
\section*{Appendix --- Reflection}

The information in this section will be used to evaluate the team members on the
graduate attribute of Lifelong Learning.  Please answer the following questions:

\begin{enumerate}
  \item 
  \item 
\end{enumerate}

\end{document}