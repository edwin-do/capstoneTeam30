\documentclass[12pt, titlepage]{article}

\usepackage[shortlabels]{enumitem}
\usepackage{comment}
\usepackage{booktabs}
\usepackage{tabularx}
\usepackage{hyperref}
\usepackage{float}
\usepackage{soul}
\usepackage{changepage}
\usepackage{graphicx}
\hypersetup{
    colorlinks,
    citecolor=black,
    filecolor=black,
    linkcolor=red,
    urlcolor=blue
}
\usepackage[round]{natbib}

%% Comments

\usepackage{color}

\newif\ifcomments\commentstrue %displays comments
%\newif\ifcomments\commentsfalse %so that comments do not display

\ifcomments
\newcommand{\authornote}[3]{\textcolor{#1}{[#3 ---#2]}}
\newcommand{\todo}[1]{\textcolor{red}{[TODO: #1]}}
\else
\newcommand{\authornote}[3]{}
\newcommand{\todo}[1]{}
\fi

\newcommand{\wss}[1]{\authornote{blue}{SS}{#1}} 
\newcommand{\plt}[1]{\authornote{magenta}{TPLT}{#1}} %For explanation of the template
\newcommand{\an}[1]{\authornote{cyan}{Author}{#1}}

%% Common Parts

\newcommand{\progname}{ProgName} % PUT YOUR PROGRAM NAME HERE
\newcommand{\authname}{Team \#, Team Name
\\ Student 1 name
\\ Student 2 name
\\ Student 3 name
\\ Student 4 name} % AUTHOR NAMES                  

\usepackage{hyperref}
    \hypersetup{colorlinks=true, linkcolor=blue, citecolor=blue, filecolor=blue,
                urlcolor=blue, unicode=false}
    \urlstyle{same}
                                


\begin{document}

\title{Verification and Validation Report: \progname} 
\author{\authname}
\date{\today}
	
\maketitle

\pagenumbering{roman}

\section{Revision History}

\begin{table}[H]
  \caption{\bf Revision History}
  \begin{tabularx}{\textwidth}{p{2.5cm}p{2.5cm}X}
  \toprule {\bf Date} & {\bf Developer} & {\bf Notes/Changes}\\
  \midrule
  Mar 7, 2023 & Abdul Nour & Updated template\\
  Mar 8, 2023 & Abdul Nour & Added Functional Requirements Evaluation \& Trace to Requirements\\
  \bottomrule
  \end{tabularx}
  \end{table}

~\newpage

\section{Symbols, Abbreviations and Acronyms}

\renewcommand{\arraystretch}{1.2}
\begin{tabular}{l l} 
  \toprule		
  \textbf{symbol} & \textbf{description}\\
  \midrule 
  T & Test\\
  \bottomrule
\end{tabular}\\

\wss{symbols, abbreviations or acronyms -- you can reference the SRS tables if needed}

\newpage

\tableofcontents

\listoftables %if appropriate

\listoffigures %if appropriate

\newpage

\pagenumbering{arabic}

%This document ...

\section{Functional Requirements Evaluation}

\noindent The following table suggests examples of system level tests that were executed in order to verify that the functional requirements of the system were met.

\begin{table}[H]
	\centering
	\caption{Test Cases for Functional Requirements}
	\label{my-label}
%	\begin{tabular}{|l|l|l|l|l|}
	\begin{tabular}{|p{0.85cm}|p{3.5cm}|p{3.5cm}|p{1.3cm}|p{3cm}|}
		\hline
		\textbf{ID} & \textbf{User Action} & \textbf{Expected Result}  & \textbf{Result}  & \textbf{Notes}  \\ \hline
		ST1 & Check "Current Source" radio button & Current Source is reset & PASS &\\ \hline
		ST2 & Type an integer (1) in the "Current Supply" textbox and click "Set" button & Current Supply is set to 1mA and displayed & PASS &\\ \hline
		ST3 & Type an integer (10) in the "Compliance" textbox and click "Set" button & Compliance voltage is set to 10V & PASS & Compliance voltage is shown on the Keithley current source display \\ \hline
		ST4 & Click "Current ON" button & Current Source is supplying current & PASS & Current Source blue LED turns on \\ \hline
		ST4 & Click "Current OFF" button & Current Source stops supplying current & PASS & Current Source blue LED turns off \\ \hline
		ST6 & Check "Nano-Voltmeter" radio button & Nano-Voltmeter is reset & PASS &\\ \hline
	\end{tabular}
\end{table}

%\newpage

\begin{table}[H]
	\centering
%	\begin{tabular}{|l|l|l|l|l|}
	\begin{tabular}{|p{0.85cm}|p{3.5cm}|p{3.5cm}|p{1.3cm}|p{3cm}|}
		\hline
		\textbf{ID} & \textbf{User Action} & \textbf{Expected Result}  & \textbf{Result}  & \textbf{Notes}  \\ \hline
		ST7 & Click "Start Capture" button & Values from the voltmeter are continuously displayed on the application along with real-time calculations & PASS & \\ \hline
		ST8 & Click "Stop Capture" button & Values and calcualtions stop generating, latest batch remains visible & PASS &\\ \hline
		ST9 & Current supply is set up and capture is started & Accurate calculations of resistivity are displayed continuously & PASS &\\ \hline
		ST10 & System is capturing values, thermal treatment is taking place & Critical changes in resistivity are noted in the capture log & FAIL & Not yet implemented, needs component added to calculation module\\ \hline
		ST11 & Choose an "Integration Rate" of (1 PLC) from the drop-down menu & Voltmeter display shows faster reading than default & PASS & Obsereved on Keithley Nano-voltmeter\\ \hline
		ST12 & System is capturing values, thermal treatment is taking place & Changes in resistivity-temperature slopes are noted in caputre log and displayed & FAIL & Not yet implemented, calculation module \& notification system additions needed\\ \hline
		ST13 & Current supply is set up and capture is started, "Stop Experiment" button is clicked on remote interface & Capture and current supply are turned off & FAIL & Remote interface is yet to be implemented\\ \hline
	\end{tabular}
\end{table}

\newpage

\section{Nonfunctional Requirements Evaluation}

\subsection{Usability}
		
\subsection{Performance}

\subsection{etc.}
	
\section{Comparison to Existing Implementation}	

This section will not be appropriate for every project.

\section{Unit Testing}

\section{Changes Due to Testing}

\wss{This section should highlight how feedback from the users and from 
the supervisor (when one exists) shaped the final product.  In particular 
the feedback from the Rev 0 demo to the supervisor (or to potential users) 
should be highlighted.}

\section{Automated Testing}
		
\section{Trace to Requirements}

\begin{table}[H]
	\centering
	\caption{Requirements Traceability}
	\label{my-label}
	\begin{tabular}{|c|c|c|}
		\hline
		\textbf{Test} & \textbf{Requirement} & \textbf{Plan} \\ \hline
		ST1 & FR1 &FR-T1 \\ \hline
		ST2 & FR1 &FR-T1\\ \hline
		ST3 & FR1 &FR-T1\\ \hline
		ST4 & FR1 &FR-T1\\ \hline
		ST5 & FR1 &FR-T1\\ \hline
		ST6 & FR1 &FR-T2\\ \hline
		ST7 & FR1 &FR-T2\\ \hline
		ST8 & FR1 &FR-T2\\ \hline
		ST9 & FR1 &FR-T3\\ \hline
		ST10 & FR2 &FR-T4\\ \hline
		ST11 & FR3 &FR-T5\\ \hline
		ST12 & FR4 &FR-T6\\ \hline
		ST13 & FR5 & FR-T7\\ \hline
	\end{tabular}
\end{table}
		
\section{Trace to Modules}		

\section{Code Coverage Metrics}

\bibliographystyle{plainnat}
%\bibliography{../../refs/References}

\newpage{}
\section*{Appendix --- Reflection}

The information in this section will be used to evaluate the team members on the
graduate attribute of Reflection.  Please answer the following question:

\begin{enumerate}
  \item In what ways was the Verification and Validation (VnV) Plan different
  from the activities that were actually conducted for VnV?  If there were
  differences, what changes required the modification in the plan?  Why did
  these changes occur?  Would you be able to anticipate these changes in future
  projects?  If there weren't any differences, how was your team able to clearly
  predict a feasible amount of effort and the right tasks needed to build the
  evidence that demonstrates the required quality?  (It is expected that most
  teams will have had to deviate from their original VnV Plan.)
\end{enumerate}

\end{document}