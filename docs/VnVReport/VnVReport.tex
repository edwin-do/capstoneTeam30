\documentclass[12pt, titlepage]{article}

\usepackage{booktabs}
\usepackage{tabularx}
\usepackage{hyperref}
\hypersetup{
    colorlinks,
    citecolor=black,
    filecolor=black,
    linkcolor=red,
    urlcolor=blue
}
\usepackage[round]{natbib}
\usepackage{multirow}

%% Comments

\usepackage{color}

\newif\ifcomments\commentstrue %displays comments
%\newif\ifcomments\commentsfalse %so that comments do not display

\ifcomments
\newcommand{\authornote}[3]{\textcolor{#1}{[#3 ---#2]}}
\newcommand{\todo}[1]{\textcolor{red}{[TODO: #1]}}
\else
\newcommand{\authornote}[3]{}
\newcommand{\todo}[1]{}
\fi

\newcommand{\wss}[1]{\authornote{blue}{SS}{#1}} 
\newcommand{\plt}[1]{\authornote{magenta}{TPLT}{#1}} %For explanation of the template
\newcommand{\an}[1]{\authornote{cyan}{Author}{#1}}

%% Common Parts

\newcommand{\progname}{ProgName} % PUT YOUR PROGRAM NAME HERE
\newcommand{\authname}{Team \#, Team Name
\\ Student 1 name
\\ Student 2 name
\\ Student 3 name
\\ Student 4 name} % AUTHOR NAMES                  

\usepackage{hyperref}
    \hypersetup{colorlinks=true, linkcolor=blue, citecolor=blue, filecolor=blue,
                urlcolor=blue, unicode=false}
    \urlstyle{same}
                                


\begin{document}

\title{Verification and Validation Report: \progname} 
\author{\authname}
\date{\today}
	
\maketitle

\pagenumbering{roman}

\section{Revision History}

\begin{tabularx}{\textwidth}{p{3cm}p{2cm}X}
\toprule {\bf Date} & {\bf Name} & {\bf Notes}\\
\midrule
Mar 8 2023 & Edwin Do & Added usability test results \\
Mar 8 2023 & Edwin Do & Added Traceability matrices \\
\bottomrule
\end{tabularx}

~\newpage

\section{Symbols, Abbreviations and Acronyms}
change capacity to size

\renewcommand{\arraystretch}{1.2}
\begin{tabular}{l l} 
  \toprule		
  \textbf{symbol} & \textbf{description}\\
  \midrule 
  T & Test\\
  \bottomrule
\end{tabular}\\

\wss{symbols, abbreviations or acronyms -- you can reference the SRS tables if needed}

\newpage

\tableofcontents

\listoftables %if appropriate

\listoffigures %if appropriate

\newpage

\pagenumbering{arabic}

This document ...
Section 11 Code Coverage Metrics is removed.

\section{Functional Requirements Evaluation}

\section{Nonfunctional Requirements Evaluation}

In the section

\pagebreak
\subsection{Usability}
The table below shows the results of our usability tests based on the tests in the V\&V plan based on the requirements mentioned in the SRS.
Each requirement can be traced to multiple unit tests and the usability survey used can be found in the Appendix.\\
\\
\begin{tabular}{ |p{2.5cm}||p{2cm}|p{4cm}|p{4cm}|p{1.5cm}| }
  \hline
  \multicolumn{5}{|c|}{Usability Tests} \\
  \hline
  Test Requirement & Related Unit Tests & Description & Expected Result & Result\\
  \hline
  NF-UT1   & AF  & Completing tasks without additional assistance. & User will be complete tasks successfully & PASS\\
  \hline
  NF-UT2   & AX  & Interact with interace to modify parameters. & User will be able to modify the parameters accurately and quickly.& PASS\\
  \hline
  NF-UT3   & AL  & Completing all tasks with limited number of mistakes & User will be able to complete all tasks with \textsl{MAX\_MISTAKE} & PASS\\
  \hline
  NF-UT3   & AL  & Verifying if interacting with the application previously improves ease of use (learnability)  & User will be able to complete the tasks more quickly and accurately the second time & PASS\\
  \hline
  NF-UT5   & AS  & Verifying how calculations are performed is hidden &  User will not know how calculations are performed after doing the set of tasks & PASS\\
  \hline
  NF-UT6   & AS  & Verifying appropriate application size upon installation&  User will install application onto computer and verify that the application size is less than or equal to \textsl{MAX\_SIZE} & PASS\\
  \hline
 \end{tabular}
		
\subsection{Performance}

\subsection{etc.}
	
\section{Comparison to Existing Implementation}	

This section will not be appropriate for every project.

\section{Unit Testing}

\subsection{Calculation Module}
\begin{tabular}{ |p{1.5cm}||p{2.5cm}|p{3cm}|p{2cm}|p{2cm}|p{1.5cm}|}
  \hline
  \multicolumn{6}{|c|}{Unit Tests for Calculation Module} \\
  \hline
  Unit Test ID & Description & Input & Expected Output & Output & Result\\
  \hline
  UT-C1   & Testing the getResistance Method  &  From HardwareInput ADT: Voltage = 5, Current = 1 & 5.000 & 5.000 & PASS \\
  \hline
  UT-C1   & Testing the getResistance Method  &  From HardwareInput ADT: Voltage = 4, Current = 0.6 & 6.667 & 6.667 & PASS \\
  \hline
  UT-C2   & Testing the getResistance Method  &  From HardwareInput ADT: Voltage = -5, Current = 1 & Invalid & Invalid & PASS \\
  \hline
  UT-C2   & Testing the getResistance Method  &  From HardwareInput ADT: Voltage = -4, Current = 0.6 & Invalid & Invalid & PASS \\
  \hline
  UT-C3   & Testing the getResistivity Method  &  Resistance = 3, Area = 2.5, Length = 2 & 3.750 & 3.750 & PASS \\
  \hline
  UT-C3   & Testing the getResistivity Method  &  Resistance = 2, Area = 2.8, Length = 1 & 5.600 & 5.600 & PASS \\
  \hline
  UT-C4   & Testing the getResistivity Method  &  Resistance = A, Area = 2.8, Length = 0 & Invalid & Invalid & PASS \\
  \hline
  UT-C4   & Testing the getResistivity Method  &  Resistance = 1, Area = 0, Length = BC & Invalid & Invalid & PASS \\
  \hline
  UT-C5  & Testing the calcResistance Method  &  Voltage = 2.6, Current = 2 & 1.300 & 1.300 & PASS \\
  \hline
  UT-C5  & Testing the calcResistance Method  &  Voltage = 3.6, Current = 2 & 1.800 & 1.800 & PASS \\
  \hline
 \end{tabular}

 \pagebreak
 \begin{tabular}{ |p{1.5cm}||p{2.5cm}|p{3cm}|p{2cm}|p{2cm}|p{1.5cm}|}
  \hline
  \multicolumn{6}{|c|}{Unit Tests for Calculation Module (Continued)} \\
  \hline
  Unit Test ID & Description & Input & Expected Output & Output & Result\\
  \hline
  UT-C6  & Testing the calcResistance Method  &  Voltage = AB, Current = 0 & Invalid & Invalid & PASS \\
  \hline
  UT-C6  & Testing the calcResistance Method  &  Voltage = 0, Current = DR & Invalid & Invalid & PASS \\
  \hline
  UT-C7  & Testing the calcResistivity Method  &  Resistance = 2, Area = 1.5, Length = 2 & 1.500 & 1.500 & PASS \\
  \hline
  UT-C7  & Testing the calcResistivity Method  &  Resistance = 1.8, Area = 1.3, Length = 1.5  & 1.560 & 1.560 & PASS \\
  \hline
  UT-C8  & Testing the calcResistivity Method  &  Resistance = AB, Area = 1, Length = 2  & Invalid & Invalid & PASS \\
  \hline
  UT-C8  & Testing the calcResistivity Method  &  Resistance = 2, Area = 2, Length = NT  & Invalid & Invalid & PASS \\
  \hline
 \end{tabular}

 \subsection{User Input Validation Module}
 \begin{tabular}{ |p{1.5cm}||p{2cm}|p{2.5cm}|p{2cm}|p{2cm}|p{1.5cm}|}
  \hline
  \multicolumn{6}{|c|}{Unit Tests for User Input Validation Module} \\
  \hline
  Unit Test ID & Description & Input & Expected Output & Output & Result\\
  \hline
  UT-UI1   & Testing the getUserInput Method  &  N/A  & UserInput (SamplingRate: 60, SampleLength: 4, SampleWidth:2) & UserInput (SamplingRate: 60, SampleLength: 4, SampleWidth:2) & PASS \\
  \hline
  UT-UI2   & Testing the validateFileData Method  &  FileName: Test, Date: 03/01/2023, Name: Test & TRUE & TRUE & PASS \\
  \hline
  UT-UI2   & Testing the validateFileData Method  &  FileName: Output, Date: 03/01/2023, Name: John & TRUE & TRUE & PASS \\
  \hline
  UT-UI3   & Testing the validateFileData Method  &  FileName: Output, Date: Someday, Name: Tester & FALSE & FALSE & PASS \\
  \hline
  UT-UI3   & Testing the validateFileData Method  &  FileName: 0, Date: 03/01/2023, Name: 0 & FALSE & FALSE & PASS \\
  \hline
 \end{tabular}

 \pagebreak 

 \begin{tabular}{ |p{1.5cm}||p{2cm}|p{2.5cm}|p{2cm}|p{2cm}|p{1.5cm}|}
  \hline
  \multicolumn{6}{|c|}{Unit Tests for User Input Validation Module (Continued)} \\
  \hline
  Unit Test ID & Description & Input & Expected Output & Output & Result\\
  \hline
  UT-UI4   & Testing the validateSampleData Method  &  SamplingRate: 50, SampleLength:2, SampleWidth:5 & TRUE & TRUE & PASS \\
  \hline
  UT-UI4   & Testing the validateSampleData Method  &  SamplingRate: 60, SampleLength:1.5, SampleWidth:3 & TRUE & TRUE & PASS \\
  \hline
  UT-UI5   & Testing the validateSampleData Method  &  SamplingRate: -7, SampleLength:1.5, SampleWidth:3 & FALSE & FALSE & PASS \\
  \hline
  UT-UI5  & Testing the validateSampleData Method  &  SamplingRate: 60, SampleLength:-5, SampleWidth:3 & FALSE & FALSE & PASS \\
  \hline
 \end{tabular}

 \subsection{Hardware Input Validation Module}
 \begin{tabular}{ |p{1.5cm}||p{2cm}|p{2cm}|p{3cm}|p{3cm}|p{1.5cm}|}
  \hline
  \multicolumn{6}{|c|}{Unit Tests for Hardware Input Validation Module} \\
  \hline
  Unit Test ID & Description & Input & Expected Output & Output & Result\\
  \hline
  UT-HI1   & Testing the getHardwareInput Method  &  -  & HardwareInput ( Voltage: 5, Current: 3) & HardwareInput ( Voltage: 5, Current: 3) & PASS \\
  \hline
  UT-HI2   & Testing the validateParameters Method  &  Voltage: 3.0 , Time: 5:31 PM , Current: 1.0 & TRUE & TRUE & PASS \\
  \hline
  UT-HI2   & Testing the validateFileData Method  &  Voltage: 3.2 , Time: 5:45 PM , Current: 1.2 & TRUE & TRUE & PASS \\
  \hline
  UT-HI3   & Testing the validateFileData Method  &  Voltage: 10 , Time: 5:48 PM , Current: NY & FALSE & FALSE & PASS \\
  \hline
  UT-HI3   & Testing the validateFileData Method  &  Voltage: YT , Time: 5:51 PM , Current: 0.5  & FALSE & FALSE & PASS \\
  \hline
 \end{tabular}

\section{Changes Due to Testing}

\section{Automated Testing}
		
\section{Trace to Requirements}
\begin{tabular}{ |p{3cm}||p{4cm}|p{4cm}|p{4cm}|p{4cm}| }
  \hline
  \multicolumn{4}{|c|}{Traceability Matrix to Non Functional Requirements} \\
  \hline
  Requirement Type & Requirement(SRS) & Test Requirement & Related Unit Tests \\
  \hline
  Non Functional   & NFR-U1  & NF-UT1 & U \\ \hline
  Non Functional   & NFR-U2  & NF-UT2 & U \\ \hline
  Non Functional   & NFR-U3  & NF-UT3 & U \\ \hline
  Non Functional   & NFR-U4  & NF-UT4 & U \\ \hline
  Non Functional   & NFR-U5  & NF-UT5 & U \\ \hline
  Non Functional   & NFR-U6  & NF-UT6 & U \\ \hline
  
 \end{tabular}

 \begin{tabular}{ |p{3cm}||p{4cm}|p{4cm}|p{4cm}|p{4cm}| }
  \hline
  \multicolumn{4}{|c|}{Traceability Matrix to Functional Requirements} \\
  \hline
  Requirement Type & Requirement(SRS) & Test Requirement & Related Unit Tests \\
  \hline
  Functional   & FR1  & FR-T1 & U \\ \hline
  Functional   & FR2  & FR-T2 & U \\ \hline
  Functional   & FR3  & FR-T3 & U \\ \hline
  Functional   & FR4  & FR-T4 & U \\ \hline
  Functional   & FR5  & FR-T5 & U \\ \hline
  Functional   & FR6  & FR-T6 & U \\ \hline
  
 \end{tabular}
		
\section{Trace to Modules}		

\bibliographystyle{plainnat}
\bibliography{../../refs/References}

\newpage{}
\section*{Appendix --- Reflection}

The information in this section will be used to evaluate the team members on the
graduate attribute of Lifelong Learning.  Please answer the following questions:

\begin{enumerate}
  \item 
  \item 
\end{enumerate}

\end{document}