\documentclass[12pt, titlepage]{article}

\usepackage{amsmath, mathtools}

\usepackage[round]{natbib}
\usepackage{amsfonts}
\usepackage{amssymb}
\usepackage{graphicx}
\usepackage{colortbl}
\usepackage{xr}
\usepackage{hyperref}
\usepackage{longtable}
\usepackage{xfrac}
\usepackage{tabularx}
\usepackage{float}
\usepackage{siunitx}
\usepackage{booktabs}
\usepackage{multirow}
\usepackage[section]{placeins}
\usepackage{caption}
\usepackage{fullpage}

\hypersetup{
bookmarks=true,     % show bookmarks bar?
colorlinks=true,       % false: boxed links; true: colored links
linkcolor=red,          % color of internal links (change box color with linkbordercolor)
citecolor=blue,      % color of links to bibliography
filecolor=magenta,  % color of file links
urlcolor=cyan          % color of external links
}

\usepackage{array}

\externaldocument{../../SRS/SRS}

%% Comments

\usepackage{color}

\newif\ifcomments\commentstrue %displays comments
%\newif\ifcomments\commentsfalse %so that comments do not display

\ifcomments
\newcommand{\authornote}[3]{\textcolor{#1}{[#3 ---#2]}}
\newcommand{\todo}[1]{\textcolor{red}{[TODO: #1]}}
\else
\newcommand{\authornote}[3]{}
\newcommand{\todo}[1]{}
\fi

\newcommand{\wss}[1]{\authornote{blue}{SS}{#1}} 
\newcommand{\plt}[1]{\authornote{magenta}{TPLT}{#1}} %For explanation of the template
\newcommand{\an}[1]{\authornote{cyan}{Author}{#1}}

%% Common Parts

\newcommand{\progname}{ProgName} % PUT YOUR PROGRAM NAME HERE
\newcommand{\authname}{Team \#, Team Name
\\ Student 1 name
\\ Student 2 name
\\ Student 3 name
\\ Student 4 name} % AUTHOR NAMES                  

\usepackage{hyperref}
    \hypersetup{colorlinks=true, linkcolor=blue, citecolor=blue, filecolor=blue,
                urlcolor=blue, unicode=false}
    \urlstyle{same}
                                


\begin{document}

\title{Module Interface Specification for \progname{}}

\author{\authname}

\date{\today}

\maketitle

\pagenumbering{roman}

\section{Revision History}

\begin{tabularx}{\textwidth}{p{3cm}p{2cm}X}
\toprule {\bf Date} & {\bf Developer} & {\bf Notes}\\
\midrule
Jan 17, 2023 & Timothy Chen & Added Modules to Module Decomposition\\
Jan 18, 2023 & Timothy Chen & Added Current State Module \\
Jan 18, 2023 & Timothy Chen & Added File Output Module \\
Jan 18, 2023 & Timothy Chen & Added Graphical Output Module \\
Jan 18 2023 & Edwin Do & Added MIS info for UserInputValidation, HardwareInputValidation, and Calculation Modules\\
Jan 18 2023 & Edwin Do & Added state invariants\\
\bottomrule
\end{tabularx}

~\newpage

\section{Symbols, Abbreviations and Acronyms}

See SRS Documentation at \href{https://github.com/edwin-do/capstoneTeam30/blob/main/docs/SRS/SRS.pdf}{here}.

% \wss{Also add any additional symbols, abbreviations or acronyms}

\newpage

\tableofcontents

\newpage

\pagenumbering{arabic}

\section{Introduction}

\noindent The following document details the Module Interface Specifications for Measuring Microstructure Changes During Thermal Treatment. This project will allow the Materials Engineering lab at McMaster University, led by Dr. Zurob, to use a software capable of providing data on thermally treated metals. The data includes measurements of resistivity of the material as well as graphical representations and analysis. \\

\noindent Complementary documents include the System Requirement Specifications and Module Guide. The full documentation and implementation can be found at our \href{https://github.com/edwin-do/capstoneTeam30}{GitHub repository}.

\section{Notation}

\wss{You should describe your notation.  You can use what is below as
  a starting point.}

The structure of the MIS for modules comes from \citet{HoffmanAndStrooper1995},
with the addition that template modules have been adapted from
\cite{GhezziEtAl2003}.  The mathematical notation comes from Chapter 3 of
\citet{HoffmanAndStrooper1995}.  For instance, the symbol := is used for a
multiple assignment statement and conditional rules follow the form $(c_1
\Rightarrow r_1 | c_2 \Rightarrow r_2 | ... | c_n \Rightarrow r_n )$.

The following table summarizes the primitive data types used by \progname. 

\begin{center}
\renewcommand{\arraystretch}{1.2}
\noindent 
\begin{tabular}{l l p{7.5cm}} 
\toprule 
\textbf{Data Type} & \textbf{Notation} & \textbf{Description}\\ 
\midrule
character & char & a single symbol or digit\\
integer & $\mathbb{Z}$ & a number without a fractional component in (-$\infty$, $\infty$) \\
natural number & $\mathbb{N}$ & a number without a fractional component in [1, $\infty$) \\
real & $\mathbb{R}$ & any number in (-$\infty$, $\infty$)\\
\bottomrule
\end{tabular} 
\end{center}

\noindent
The specification of \progname \ uses some derived data types: sequences, strings, and
tuples. Sequences are lists filled with elements of the same data type. Strings
are sequences of characters. Tuples contain a list of values, potentially of
different types. In addition, \progname \ uses functions, which
are defined by the data types of their inputs and outputs. Local functions are
described by giving their type signature followed by their specification.

\section{Module Decomposition}

The following table is taken directly from the Module Guide document for this project.

\begin{table}[h!]
\centering
\begin{tabular}{p{0.3\textwidth} p{0.6\textwidth}}
\toprule
\textbf{Level 1} & \textbf{Level 2}\\
\midrule

{Hardware-Hiding} & ~ \\
\midrule

\multirow{7}{0.3\textwidth}{Behaviour-Hiding}
& Input Communication Module (\ref{IC})\\
& Output Communication Module (\ref{OC})\\
& Remote Access Module (\ref{RA})\\
& Current State Module (\ref{CS})\\ 
& FileOutput Module (\ref{FO})\\
& Graphical Output Module (\ref{GO})\\
\midrule

\multirow{3}{0.3\textwidth}{Software Decision} 
& Calculation Module (\ref{C})\\
& User Input Validation Module (\ref{UI})\\
& Hardware Input Validation Module (\ref{HI})\\
\bottomrule

\end{tabular}
\caption{Module Hierarchy}
\label{TblMH}
\end{table}

\newpage

\section{MIS of Current State Module} \label{CS} 

\subsection{Module}

Current State Module

\subsection{Uses}

\subsubsection{Imported Types}

HardwareInput: ( $Voltage: real $ ; $Time: real$; $Temperature: real$; $Current: real$ )\\
UserInput: ( $SamplingRate: real$; $SampleLengthgth: real$; $SampleWidth: real$; $Filename: string$; $Name: string$; $SampleName: string$; $Date: string$)

\subsubsection{Imported Access Programs}

GetUserInput(): UserInput\\
GetHardwareInput(): HardwareInput\\ 


\subsection{Syntax}

\subsubsection{Exported Constants}

N/A
\subsubsection{Exported Access Programs}

\begin{center}
\begin{tabular}{p{5cm} p{5cm} p{1cm} p{3cm}}
\hline
\textbf{Name} & \textbf{In} & \textbf{Out} & \textbf{Exceptions} \\
\hline
StateInit() & &  &  \\
DisplayUserInfo() & string, string, string, string, real, real, real &  & INVALID \\
DisplayHardwareState() & real, real, real, real&  & INVALID \\
\hline
\end{tabular}
\end{center}

\subsection{Semantics}

\subsubsection{State Variables}

N/A

\subsubsection{State Invariant}

N/A

\subsubsection{Environment Variables}

ApplicationWindow: the screen inferface where the information displayed to the user

\subsubsection{Assumptions}

StateInit() is called before any other access program

\subsubsection{Access Routine Semantics}

\noindent StateInit():
\begin{itemize}
\item transition: State Display is initialized on ApplicationWindow
\item exception: none
\end{itemize}

\noindent DisplayUserInfo(Name, SampleName, Date, Filename, SamplingRate, SampleLength, SampleWidth):
\begin{itemize}
\item transition: Display Name, SampleName Date, Filename, SamplingRate, SampleLength, and SampleWidth on the ApplicationWindow
\item exception: $exc:= SamplingRate \notin \mathbb{R} \lor SamplingRate < 0 \lor SampleLength \notin \mathbb{R} \lor SampleLength < 0 \lor SampleWidth \notin \mathbb{R} \lor SampleWidth < 0 \Rightarrow INVALID$
\end{itemize}

\noindent DisplayHardwareState(Voltage, Current, Time, Temperature):
\begin{itemize}
\item transition: Display Voltage, Current, and Time on the ApplicationWindow
\item exception: $exc:= Voltage \notin  \mathbb{R} \lor Voltage < 0 \lor  Current \notin  \mathbb{R} \lor Current < 0 \lor  Time \notin  \mathbb{R} \lor Time < 0 \lor  Temperature \notin  \mathbb{R} \Rightarrow INVALID$
\end{itemize}

\subsubsection{Local Functions}
N/A

\newpage

\section{MIS of FileOutput Module} \label{FO} 

\subsection{Module}

FileOutput Module

\subsection{Uses}

\subsubsection{Imported Types}

HardwareInput: ( $Voltage: real $ ; $Time: real$; $Temperature: real$; $Current: real$ )\\
UserInput: ( $SamplingRate: real$; $SampleLength: real$; $SampleWidth: real$; $Filename: string$ $Name: string$; $SampleName: string$; $Date: string$)

\subsubsection{Imported Access Programs}
  GetResistivity(): Real\\
  GetResistance(): Real\\
  GetUserInput(): UserInput\\
  GetHardwareInput(): HardwareInput

\subsection{Syntax}

\subsubsection{Exported Constants}

N/A

\subsubsection{Exported Access Programs}

\begin{center}
\begin{tabular}{p{4cm} p{6cm} p{2cm} p{3cm}}
\hline
\textbf{Name} & \textbf{In} & \textbf{Out} & \textbf{Exceptions} \\
\hline
FileInit() & &  &  \\
WriteUserInput() & string, string, string, real real, real&  & INVALID \\
WriteSampleOutput()& real, real, real, real, real, real & record & INVALID \\

\hline
\end{tabular}
\end{center}

\subsection{Semantics}

\subsubsection{State Variables}

N/A
\subsubsection{State Invariant}

N/A
\subsubsection{Environment Variables}

OutputFile: a file used to store data such as the user inputs and hardware outputs

\subsubsection{Assumptions}

FileInit() is called before any other access program. 

\subsubsection{Access Routine Semantics}

\noindent FileInit():
\begin{itemize}
\item transition: Initializes an empty file
\item exception: none
\end{itemize}

\noindent WriteUserInput(Name, SampleName, Date, SamplingRate, SampleLength, SampleWidth):
\begin{itemize}
\item transition: Write user input into the first line of the OutputFile 
\item exception: $exc:= SamplingRate \notin \mathbb{R} \lor SamplingRate < 0 \lor SampleLength \notin \mathbb{R} \lor SampleLength < 0 lor SampleWidth \notin \mathbb{R} \lor SampleWidth < 0 \Rightarrow INVALID$
\end{itemize}

\noindent WriteSampleOutput(Time, Temperature, Voltage, Current, Resistance, Resistivity):
\begin{itemize}
\item transition: Write each data set into the OutputFile at each time interval
\item exception: $exc:= Time \notin  \mathbb{R} \lor Time < 0 \lor  Temperature \notin  \mathbb{R} \lor Voltage < 0 \lor  Voltage \notin  \mathbb{R} \lor Current < 0 \lor  Current \notin  \mathbb{R} \lor Resistance < 0 \lor  Resistance \notin  \mathbb{R} \lor Resistivity < 0\lor  Resistance \notin  \mathbb{R} \lor Resistivity < 0\Rightarrow INVALID$
\end{itemize}

\subsubsection{Local Functions}

N/A

\newpage

\section{MIS of Graphical Output Module} \label{GO} 

\subsection{Module}

File Output Module

\subsection{Uses}

\subsubsection{Imported Types}

HardwareInput: ( $Voltage: real $ ; $Time: real$; $Temperature: real$; $Current: real$ )\\

\subsubsection{Imported Access Programs}

  GetResistivity(): Real
  GetResistance(): Real
  GetHardwareInput(): HardwareInput

\subsection{Syntax}

\subsubsection{Exported Constants}

\subsubsection{Exported Access Programs}

\begin{center}
\begin{tabular}{p{6cm} p{2cm} p{2cm} p{3cm}}
\hline
\textbf{Name} & \textbf{In} & \textbf{Out} & \textbf{Exceptions} \\
\hline
GraphInit() & &  &  \\
GraphTimeVResistance() & real, real &  & INVALID \\
GraphTimeVResistivity() & real, real &  & INVALID \\
GraphVoltageVResistence() & real, real &  & INVALID \\
GraphVoltageVResistivity() & real, real &  & INVALID \\
GraphTemperatureVResistence() & real, real &  & INVALID \\
GraphTemperatureVResistivity() & real, real &  & INVALID \\
\hline
\end{tabular}
\end{center}

\subsection{Semantics}

\subsubsection{State Variables}

N/A

\subsubsection{State Invariant}

N/A
\subsubsection{Environment Variables}

ApplicationWindow: the screen inferface where the information displayed to the user

\subsubsection{Assumptions}

GraphInit() is called before any other access program

\subsubsection{Access Routine Semantics}

\noindent GraphInit():
\begin{itemize}
\item transition: Graph is initialized on ApplicationWindow
\item exception: none
\end{itemize}

\noindent GraphTimeVResistance(Time, Resistance):
\begin{itemize}
\item transition: Disaply graph of Time versus Resistance on ApplicationWindow 
\item exception: $exc:= Time \notin  \mathbb{R} \lor Time < 0 \lor  Resistance \notin  \mathbb{R} \lor Resistance < 0\Rightarrow INVALID$
\end{itemize}

\noindent GraphTimeVResistivity(Time, Resistivity):
\begin{itemize}
\item transition: Display graph of Time versus Resistivity on ApplicationWindow
\item exception: $exc:= Time \notin  \mathbb{R} \lor Time < 0 \lor  Resistivity \notin  \mathbb{R} \lor Resistivity < 0\Rightarrow INVALID$
\end{itemize}

\noindent GraphVoltageVResistance(Voltage, Resistance):
\begin{itemize}
\item transition: Display graph of Voltage versus Resistance on ApplicationWindow
\item exception: $exc:= Voltage\notin  \mathbb{R} \lor Voltage < 0 \lor  Resistance \notin  \mathbb{R} \lor Resistance < 0\Rightarrow INVALID$
\end{itemize}

\noindent GraphVoltageVResistivity(Voltage, Resistivity):
\begin{itemize}
\item transition: Display graph of Voltage versus Resistivity on ApplicationWindow
\item exception: $exc:= Voltage \notin  \mathbb{R} \lor Voltage < 0 \lor  Resistivity \notin  \mathbb{R} \lor Resistivity< 0\Rightarrow INVALID$
\end{itemize}

\noindent GraphTemperatureVResistance(Temperature, Resistance):
\begin{itemize}
\item transition: Display graph of Temperature versus Resistance on ApplicationWindow
\item exception: $exc:= Temperature \notin  \mathbb{R} \lor  Resistance \notin  \mathbb{R} \lor Resistance < 0\Rightarrow INVALID$
\end{itemize}

\noindent GraphTemperatureVResistivity(Temperature, Resistivity):
\begin{itemize}
\item transition: Display graph of Temperature versus Resistivity on ApplicationWindow
\item exception: $exc:= Temperature \notin  \mathbb{R} <  Resistivity \notin  \mathbb{R} \lor Resistivity < 0\Rightarrow INVALID$
\end{itemize}

\subsubsection{Local Functions}

N/A

\newpage


\section{MIS of Calculation Module} \label{Module} 

\subsection{Module}

Calculation

\subsection{Uses}

\subsubsection{Imported Types}
HardwareInput: \\
( $Voltage: real $ ; $Time: real$; $Temperature: real$; $Current: real$ )\\

\noindent{UserInput:} \\
( $SamplingRate: real$; $SampleLength: real$; $SampleWidth: real$; $Filename: string$; $Name: string$; $SampleName: string$; $Date: string$)

\subsubsection{Imported Access Programs}
getHardwareInput(): HardwareInput \\
getUserInput(): UserInput

\subsection{Syntax}

\subsubsection{Exported Constants}
N/A

\subsubsection{Exported Access Programs}

\begin{center}
\begin{tabular}{p{4cm} p{2cm} p{4cm} p{2cm}}
\hline
\textbf{Name} & \textbf{In} & \textbf{Out} & \textbf{Exceptions} \\
\hline
getResistance() & - & Real & INVALID \\
getResistivity() & - & Real & INVALID \\
\hline
\end{tabular}
\end{center}

\subsection{Semantics}

\subsubsection{State Variables}

Resistance : The calculated resistance value (real)\\
Resistivity : The calculated resistivity value (real) \\
SampleArea : The calculated area of sample based on the length and width from user's input

\subsubsection{State Invariants}

Resistance $\geq 0$ \\
Resistivity $\geq 0$ \\
SampleArea $\geq 0$

\subsubsection{Environment Variables}

N/A

\subsubsection{Assumptions}

We assume that the user may enter invalid values for inputs such as characters, empty spaces etc..
This will cause the program to throw and INVALID exception. This type of programmer error is also 
captured in the UserInputValidation Module to improve redundancy.

\subsubsection{Access Routine Semantics}

\noindent getResistance():
\begin{itemize}
\item transition: N/A
\item output: out:= Resistance
\item exception: $exc:= Resistance \notin \mathbb{R} \lor Resistance < 0  \Rightarrow INVALID$
\end{itemize}

\noindent getResistivity():
\begin{itemize}
\item transition: N/A
\item output: out := Resistivity
\item exception: $exc:= Resistivity \notin \mathbb{R} \lor Resistivity < 0  \Rightarrow INVALID$
\end{itemize}


\subsubsection{Local Functions}

findSampleArea(SampleLength, SampleWidth)
\begin{itemize}
  \item transition: SampleArea := SampleLength x SampleWidth
  \end{itemize}

  \noindent calcResistance(voltage, current):
  \begin{itemize}
  \item transition: Resistance := voltage/current 
  \item exception: $exc:= \\ 
  voltage \notin \mathbb{R} \lor voltage < 0 \\ 
  \lor current \notin \mathbb{R} \lor current < 0\\
   \Rightarrow INVALID$
  \end{itemize}
  
  \noindent calcResistivity(Resistance, SampleArea, SampleLength):
  \begin{itemize}
  \item transition: Resistivity := (resistance x SampleArea)/SampleLength
  \item exception: $exc:= \\ 
  voltage \notin \mathbb{R} \lor voltage < 0 \\ 
  \lor current \notin \mathbb{R} \lor current < 0\\
   \Rightarrow INVALID$
  \end{itemize}
\newpage

\section{MIS of UserInputValidation Module} \label{Module} 



\subsection{Module}

UserInputValidation

\subsection{Uses}

\subsubsection{Imported Types}

\noindent{UserInput:} \\
( $SamplingRate: real$; $SampleLength: real$; $SampleWidth: real$; $Filename: string$; $Name: string$; $SampleName: string$; $Date: string$)

\subsection{Syntax}

\subsubsection{Exported Constants}
N/A

\subsubsection{Exported Access Programs}

\begin{center}
\begin{tabular}{p{4cm} p{2cm} p{6cm} p{2cm}}
\hline
\textbf{Name} & \textbf{In} & \textbf{Out} & \textbf{Exceptions} \\
\hline
getUserInput() & - & ADT (UserInput) & INVALID \\
\hline
\end{tabular}
\end{center}

\subsection{Semantics}

\subsubsection{State Variables}
N/A

\subsubsection{State Invariants}
N/A

\subsubsection{Environment Variables}

N/A

\subsubsection{Assumptions}
We assume that the user may enter invalid values for inputs such as characters, empty spaces etc..
This will cause the program to throw and INVALID exception.

\subsubsection{Access Routine Semantics}

\noindent getUserInput():
\begin{itemize}
\item output: out:= UserInput
\item exception: $exc:= 
validateFileData \ne TRUE  \lor\\
validateSampleData \ne TRUE \\
\Rightarrow INVALID$
\end{itemize}


\subsubsection{Local Functions}
\noindent valdiateFileData(Filename, Date, Name):
\begin{itemize}
\item output: out:= TRUE
\item exception: $exc:= 
Filename.type \ne STRING \lor \\
Date.type \ne STRING  \lor \\
Name.type \ne STRING  \\
\Rightarrow INVALID$
\end{itemize}

\noindent validateSampleData(SamplingRate, SampleLength, SampleWidth):
\begin{itemize}
\item output: out:= TRUE
\item exception: $exc:= 
SamplingRate \notin \mathbb{R} \lor SamplingRate < 0 \lor \\
SampleLength \notin \mathbb{R} \lor SampleLength < 0  \lor \\
SampleWidth \notin \mathbb{R} \lor SampleWidth < 0  \\
\Rightarrow INVALID$
\end{itemize}
\newpage


\section{MIS of HardwareInputValidation Module} \label{Module}
\subsection{Module}

HardwareInputValidation

\subsection{Uses}

\subsubsection{Imported Types}
HardwareInput: \\
( $Voltage: real $ ; $Time: real$; $Temperature: real$; $Current: real$ )

\subsection{Syntax}

\subsubsection{Exported Constants}
N/A

\subsubsection{Exported Access Programs}

\begin{center}
\begin{tabular}{p{4cm} p{2cm} p{6cm} p{2cm}}
\hline
\textbf{Name} & \textbf{In} & \textbf{Out} & \textbf{Exceptions} \\
\hline
getHardwareInput() & - & ADT (HardwareInput) & INVALID \\
\hline
\end{tabular}
\end{center}

\subsection{Semantics}

\subsubsection{State Variables}
N/A

\subsubsection{State Invariants}
N/A

\subsubsection{Environment Variables}

N/A

\subsubsection{Assumptions}
N/A

\subsubsection{Access Routine Semantics}

\noindent getHardwareInput():
\begin{itemize}
\item output: out:= HardwareInput
\item exception: $exc:= 
validateParameters \ne TRUE \
\Rightarrow INVALID$
\end{itemize}


\subsubsection{Local Functions}
\noindent validateParameters(Voltage, Time, Current):
\begin{itemize}
\item output: out:= TRUE
\item exception: $exc:= 
Voltage < 0 \lor \\
Time < 0 \lor Current < 0 \\
\Rightarrow INVALID$
\end{itemize}
\newpage

\bibliographystyle {plainnat}
\bibliography {../../../refs/References}

\newpage

\section{Appendix} \label{Appendix}

\wss{Extra information if required}

\end{document}