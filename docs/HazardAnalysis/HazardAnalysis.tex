\documentclass{article}

\usepackage{booktabs}
\usepackage{tabularx}
\usepackage{hyperref}

\hypersetup{
    colorlinks=true,       % false: boxed links; true: colored links
    linkcolor=red,          % color of internal links (change box color with linkbordercolor)
    citecolor=green,        % color of links to bibliography
    filecolor=magenta,      % color of file links
    urlcolor=cyan           % color of external links
}

\title{Hazard Analysis\\\progname}

\author{\authname}

\date{}

%% Comments

\usepackage{color}

\newif\ifcomments\commentstrue %displays comments
%\newif\ifcomments\commentsfalse %so that comments do not display

\ifcomments
\newcommand{\authornote}[3]{\textcolor{#1}{[#3 ---#2]}}
\newcommand{\todo}[1]{\textcolor{red}{[TODO: #1]}}
\else
\newcommand{\authornote}[3]{}
\newcommand{\todo}[1]{}
\fi

\newcommand{\wss}[1]{\authornote{blue}{SS}{#1}} 
\newcommand{\plt}[1]{\authornote{magenta}{TPLT}{#1}} %For explanation of the template
\newcommand{\an}[1]{\authornote{cyan}{Author}{#1}}

%% Common Parts

\newcommand{\progname}{ProgName} % PUT YOUR PROGRAM NAME HERE
\newcommand{\authname}{Team \#, Team Name
\\ Student 1 name
\\ Student 2 name
\\ Student 3 name
\\ Student 4 name} % AUTHOR NAMES                  

\usepackage{hyperref}
    \hypersetup{colorlinks=true, linkcolor=blue, citecolor=blue, filecolor=blue,
                urlcolor=blue, unicode=false}
    \urlstyle{same}
                                


\begin{document}

\maketitle
\thispagestyle{empty}

~\newpage

\pagenumbering{roman}

\begin{table}[hp]
\caption{Revision History} \label{TblRevisionHistory}
\begin{tabularx}{\textwidth}{llX}
\toprule
\textbf{Date} & \textbf{Developer(s)} & \textbf{Change}\\
\midrule
Oct 13, 2022 & Abdul Nour Seddiki & Integrated the Template + Added System Boundaries and Components\\
Oct 14, 2022 & Edwin Do & Add introduction, scope and purpose of HA \\
... & ... & ...\\
\bottomrule
\end{tabularx}
\end{table}

~\newpage

\tableofcontents

~\newpage

\pagenumbering{arabic}

% \wss{You are free to modify this template.}

\section{Introduction}
This document is the hazard analysis for Team 30 (ReSprint) Capstone project. 
This project collaborates with the Department of Materials Engineering to measure the microstructure changes of samples during thermal treatment.\\

\noindent For this document, a hazard will be defined as 
a scenario where the conditions of a system and environment constitutes the potential of harm to someone or something.

\section{Scope and Purpose of Hazard Analysis}
The purpose of this hazard analysis is to document any potential unsafe behaviour in this project. 
This includes risks in this project that can arise from using either the software or hardware components.
For each hazard identified, this document will outline one or more methods to help mitigate the hazard.\\

\noindent The scope will be limited to identifying the possible hazards from the system's components, methods of mitigation, 
potential causes, and any resulting security or safety requirements.

\section{System Boundaries and Components}

\noindent This hazard analysis is conducted on the system that consists of the following components:

\begin{enumerate}
  \item Thermally treated samples
  \item The current source
  \item A thermometer
  \item The nano-voltmeter
  \item Interfaces between above devices and control computer
  \item The control computer
  \item The software application that will be installed on the control computer
\end{enumerate}

\noindent These components comprise the system in question. And they each are also considered the boundaries for this system. Some of the components mentioned are not controllable by ReSprint team, such as the thermally treated samples and all of the measurement devices and hardware including the current source, the thermometer, the nano-voltmeter, the communication interfaces and the control computer. Therefore, the only component controllable by ReSprint team is the software application and its sub-systems.

\section{Critical Assumptions}

\wss{These assumptions that are made about the software or system.  You should
minimize the number of assumptions that remove potential hazards.  For instance,
you could assume a part will never fail, but it is generally better to include
this potential failure mode.}

\section{Failure Mode and Effect Analysis}

\wss{Include your FMEA table here}

\section{Safety and Security Requirements}

\wss{Newly discovered requirements.  These should also be added to the SRS.  (A
rationale design process how and why to fake it.)}

\section{Roadmap}

\wss{Which safety requirements will be implemented as part of the capstone timeline?
Which requirements will be implemented in the future?}

\end{document}