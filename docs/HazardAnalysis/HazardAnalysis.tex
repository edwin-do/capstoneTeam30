\documentclass{article}

\usepackage{booktabs}
\usepackage{tabularx}
\usepackage{hyperref}

\hypersetup{
    colorlinks=true,       % false: boxed links; true: colored links
    linkcolor=red,          % color of internal links (change box color with linkbordercolor)
    citecolor=green,        % color of links to bibliography
    filecolor=magenta,      % color of file links
    urlcolor=cyan           % color of external links
}

\title{Hazard Analysis\\\progname}

\author{\authname}

\date{}

\input{../Comments}
%% Common Parts

\newcommand{\progname}{Measuring Microstructure Changes During Thermal Treatment} % PUT YOUR PROGRAM NAME HERE
\newcommand{\authname}{Team \#30, ReSprint
\\ Edwin Do
\\ Joseph Braun
\\ Timothy Chen
\\ Abdul Nour Seddiki
\\ Tyler Magarelli
} % AUTHOR NAMES                  

\usepackage{hyperref}
    \hypersetup{colorlinks=true, linkcolor=blue, citecolor=blue, filecolor=blue,
                urlcolor=blue, unicode=false}
    \urlstyle{same}
                                


\begin{document}

\maketitle
\thispagestyle{empty}

~\newpage

\pagenumbering{roman}

\begin{table}[hp]
\caption{Revision History} \label{TblRevisionHistory}
\begin{tabularx}{\textwidth}{llX}
\toprule
\textbf{Date} & \textbf{Developer(s)} & \textbf{Change}\\
\midrule
Oct 13, 2022 & Abdul Nour Seddiki & Integrated the Template + Added System Boundaries and Components\\
Oct 19, 2022 & Edwin Do & Add roadmap \\
... & ... & ...\\
\bottomrule
\end{tabularx}
\end{table}

~\newpage

\tableofcontents

~\newpage

\pagenumbering{arabic}

\wss{You are free to modify this template.}

\section{Introduction}

\wss{You can include your definition of what a hazard is here.}

\section{Scope and Purpose of Hazard Analysis}

\section{System Boundaries and Components}

\noindent This hazard analysis is conducted on the system that consists of the following components:

\begin{enumerate}
  \item Thermally treated samples
  \item The current source
  \item A thermometer
  \item The nano-voltmeter
  \item Interfaces between above devices and control computer
  \item The control computer
  \item The software application that will be installed on the control computer
\end{enumerate}

\noindent These components comprise the system in question. And they each are also considered the boundaries for this system. Some of the components mentioned are not controllable by ReSprint team, such as the thermally treated samples and all of the measurement devices and hardware including the current source, the thermometer, the nano-voltmeter, the communication interfaces and the control computer. Therefore, the only component controllable by ReSprint team is the software application and its sub-systems.

\section{Critical Assumptions}

\noindent The following is a list of assumptions to protect ourselves during the development of ReSprint from unforeseen hazards:

\begin{itemize}
  \item Thermal treated samples will be contained in a safe area away from the control computer and operator. 
  \item Curret source device will be used as intended and will not be misuse by the operator.
  \item Wires will not come loose during operation by the operator.
  \item Data collected from the samples will be saved correctly on the control device.
  \item Plugs and wires are attached correctly into the devices and control computer.
\end{itemize}

\section{Failure Mode and Effect Analysis}

\wss{Include your FMEA table here}

\section{Safety and Security Requirements}

\wss{Newly discovered requirements.  These should also be added to the SRS.  (A
rationale design process how and why to fake it.)}

\section{Roadmap}

As part of this project, the safety requirements that we will address
includes not using any colours and/or graphics that may cause harm or discomfort to the users
and that the application along with its equipments will be functional.
In addition, hardware-related requirements mentioned in the table above will be addressed up to the day of the final deliverable for the scope of this project.
Any other requirement may be included in the Proof of concept or the final deliverable 
which are on November 14 2023 and March 20 2023 respectively. \\

\noindent Certain requirements will not be included as part of the capstone timeline.
These requirements include ensuring that any future users will have sufficient training before interacting with the project
and that the app should not be modified by an unauthorized user. 
They should be implemented in the future by whoever oversees the use of this capstone.



% \wss{Which safety requirements will be implemented as part of the capstone timeline?
% Which requirements will be implemented in the future?}



\end{document}