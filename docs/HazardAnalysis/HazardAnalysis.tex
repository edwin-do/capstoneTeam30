\documentclass{article}

\usepackage{booktabs}
\usepackage{tabularx}
\usepackage{hyperref}

\hypersetup{
    colorlinks=true,       % false: boxed links; true: colored links
    linkcolor=red,          % color of internal links (change box color with linkbordercolor)
    citecolor=green,        % color of links to bibliography
    filecolor=magenta,      % color of file links
    urlcolor=cyan           % color of external links
}

\title{Hazard Analysis\\\progname}

\author{\authname}

\date{}

%% Comments

\usepackage{color}

\newif\ifcomments\commentstrue %displays comments
%\newif\ifcomments\commentsfalse %so that comments do not display

\ifcomments
\newcommand{\authornote}[3]{\textcolor{#1}{[#3 ---#2]}}
\newcommand{\todo}[1]{\textcolor{red}{[TODO: #1]}}
\else
\newcommand{\authornote}[3]{}
\newcommand{\todo}[1]{}
\fi

\newcommand{\wss}[1]{\authornote{blue}{SS}{#1}} 
\newcommand{\plt}[1]{\authornote{magenta}{TPLT}{#1}} %For explanation of the template
\newcommand{\an}[1]{\authornote{cyan}{Author}{#1}}

%% Common Parts

\newcommand{\progname}{ProgName} % PUT YOUR PROGRAM NAME HERE
\newcommand{\authname}{Team \#, Team Name
\\ Student 1 name
\\ Student 2 name
\\ Student 3 name
\\ Student 4 name} % AUTHOR NAMES                  

\usepackage{hyperref}
    \hypersetup{colorlinks=true, linkcolor=blue, citecolor=blue, filecolor=blue,
                urlcolor=blue, unicode=false}
    \urlstyle{same}
                                


\begin{document}

\maketitle
\thispagestyle{empty}

~\newpage

\pagenumbering{roman}

\begin{table}[hp]
\caption{Revision History} \label{TblRevisionHistory}
\begin{tabularx}{\textwidth}{llX}
\toprule
\textbf{Date} & \textbf{Developer(s)} & \textbf{Change}\\
\midrule
Oct 13, 2022 & Abdul Nour Seddiki & Integrated the Template + Added System Boundaries and Components\\
... & ... & ...\\
\bottomrule
\end{tabularx}
\end{table}

~\newpage

\tableofcontents

~\newpage

\pagenumbering{arabic}

\wss{You are free to modify this template.}

\section{Introduction}

\wss{You can include your definition of what a hazard is here.}

\section{Scope and Purpose of Hazard Analysis}

\section{System Boundaries and Components}

\noindent This hazard analysis is conducted on the system that consists of the following components:

\begin{enumerate}
  \item Thermally treated samples
  \item The current source
  \item A thermometer
  \item The nano-voltmeter
  \item Interfaces between above devices and control computer
  \item The control computer
  \item The software application that will be installed on the control computer
\end{enumerate}

\noindent These components comprise the system in question. And they each are also considered the boundaries for this system. Some of the components mentioned are not controllable by ReSprint team, such as the thermally treated samples and all of the measurement devices and hardware including the current source, the thermometer, the nano-voltmeter, the communication interfaces and the control computer. Therefore, the only component controllable by ReSprint team is the software application and its sub-systems.

\section{Critical Assumptions}

\noindent The following is a list of assumptions to protect ourselves during the development of ReSprint from unforeseen hazards:

\begin{itemize}
  \item Thermal treated samples will be contained in a safe area away from the control computer and operator. 
  \item Curret source device will be used as intended and will not be misuse by the operator.
  \item Wires will not come loose during operation by the operator.
  \item Data collected from the samples will be saved correctly on the control device.
  \item Plugs and wires are attached correctly into the devices and control computer.
\end{itemize}

\section{Failure Mode and Effect Analysis}

\wss{Include your FMEA table here}

\section{Safety and Security Requirements}

\subsection*{Safety Requirements}
\begin{enumerate}
  \item[SFR-1.] Graphics shall avoid changing of brightness at rapid rate to take account for users prone to seizures.\\
  \item[SFR-2.] Colours should avoid brightness that can be damaging to users' eyes.\\
  \item[SFR-3.] Untrained users should not need to interact with any electronic equipment to avoid potential injury.\\
\end{enumerate}

\subsection*{Security Requirements}
\begin{enumerate}
  \item[SCR-1.] Interface shall prevent any modifications or injections of measurements from the user.
  \item[SCR-2.] Only authorized users are allowed to mnodify concealed calculations, settings and/or parameters.
\end{enumerate}

\section{Roadmap}

\wss{Which safety requirements will be implemented as part of the capstone timeline?
Which requirements will be implemented in the future?}

\end{document}