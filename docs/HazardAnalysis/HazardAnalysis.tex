\documentclass{article}

\usepackage{booktabs}
\usepackage{tabularx}
\usepackage{hyperref}
\usepackage{pdflscape}
\usepackage{multirow}
\usepackage[table]{xcolor}
\usepackage{array,ragged2e}
\newcolumntype{P}[1]{>{\RaggedRight\arraybackslash}p{#1}}

\hypersetup{
    colorlinks=true,       % false: boxed links; true: colored links
    linkcolor=red,          % color of internal links (change box color with linkbordercolor)
    citecolor=green,        % color of links to bibliography
    filecolor=magenta,      % color of file links
    urlcolor=cyan           % color of external links
}

\title{Hazard Analysis\\\progname}

\author{\authname}

\date{}

%% Comments

\usepackage{color}

\newif\ifcomments\commentstrue %displays comments
%\newif\ifcomments\commentsfalse %so that comments do not display

\ifcomments
\newcommand{\authornote}[3]{\textcolor{#1}{[#3 ---#2]}}
\newcommand{\todo}[1]{\textcolor{red}{[TODO: #1]}}
\else
\newcommand{\authornote}[3]{}
\newcommand{\todo}[1]{}
\fi

\newcommand{\wss}[1]{\authornote{blue}{SS}{#1}} 
\newcommand{\plt}[1]{\authornote{magenta}{TPLT}{#1}} %For explanation of the template
\newcommand{\an}[1]{\authornote{cyan}{Author}{#1}}

%% Common Parts

\newcommand{\progname}{ProgName} % PUT YOUR PROGRAM NAME HERE
\newcommand{\authname}{Team \#, Team Name
\\ Student 1 name
\\ Student 2 name
\\ Student 3 name
\\ Student 4 name} % AUTHOR NAMES                  

\usepackage{hyperref}
    \hypersetup{colorlinks=true, linkcolor=blue, citecolor=blue, filecolor=blue,
                urlcolor=blue, unicode=false}
    \urlstyle{same}
                                


\begin{document}

\maketitle
\thispagestyle{empty}

~\newpage

\pagenumbering{roman}

\begin{table}[hp]
\caption{Revision History} \label{TblRevisionHistory}
\begin{tabularx}{\textwidth}{llX}
\toprule
\textbf{Date} & \textbf{Developer(s)} & \textbf{Change}\\
\midrule
Oct 13, 2022 & Abdul Nour Seddiki & Integrated the Template + Added System Boundaries and Components\\
... & ... & ...\\
\bottomrule
\end{tabularx}
\end{table}

~\newpage

\tableofcontents

~\newpage

\pagenumbering{arabic}

\wss{You are free to modify this template.}

\section{Introduction}

\wss{You can include your definition of what a hazard is here.}

\section{Scope and Purpose of Hazard Analysis}

\section{System Boundaries and Components}

\noindent This hazard analysis is conducted on the system that consists of the following components:

\begin{enumerate}
  \item Thermally treated samples
  \item The current source
  \item A thermometer
  \item The nano-voltmeter
  \item Interfaces between above devices and control computer
  \item The control computer
  \item The software application that will be installed on the control computer
\end{enumerate}

\noindent These components comprise the system in question. And they each are also considered the boundaries for this system. Some of the components mentioned are not controllable by ReSprint team, such as the thermally treated samples and all of the measurement devices and hardware including the current source, the thermometer, the nano-voltmeter, the communication interfaces and the control computer. Therefore, the only component controllable by ReSprint team is the software application and its sub-systems.

\section{Critical Assumptions}

\wss{These assumptions that are made about the software or system.  You should
minimize the number of assumptions that remove potential hazards.  For instance,
you could assume a part will never fail, but it is generally better to include
this potential failure mode.}


\section{Failure Mode and Effect Analysis}

\begin{landscape}
\begin{table}[h]
  \centering
  \caption{FMEA Table}
  \label{my-label}
  \begin{tabular}{|P{20mm}|P{30mm}|P{30mm}|P{30mm}|P{35mm}|P{20mm}|P{10mm}|}
    \rowcolor{gray!30}
    \hline
    \textbf{Component} & \textbf{Failure Mode} & \textbf{Effects of Failure} & \textbf{Causes of Failure} & \textbf{Recommended Action} & \textbf{Req.} & \textbf{Ref.}      \\ \hline
    Current Source
    & Current source does not provide current 
    & Nanovoltmeter cannot measure voltage across sample
    & a. Setup error \newline
      b. Hardware failure
    & a. Troubleshoot current source setup \newline
      b. Replace current source
    & a. HWR1 \newline
      b. HWR2
    & H1-1      \\ \hline
    Nanovoltmeter
    & Nanovoltmeter does not read voltage across sample
    & Voltage data cannot be communicated to App
    & a. Setup error \newline
      b. Hardware failure
    & a. Troubleshoot nanovoltmeter \newline
      b. Purchase new nanovoltmeter
    & a. HWR1 \newline
      b. HWR2
    & H1-2      \\ \hline
    Serial Connection
    & App does not receive data from hardware
    & Data cannot be communicated to App
    & a. Setup error \newline
      b. Hardware failure
    & a. Troubleshoot serial connection \newline
      b. Replace serial connection cable
    & a. HWR1 \newline
      b. HWR2
    & H1-3      \\ \hline
    \multirow{2}{*}{Windows App}
    & App is not ergonomic for user
    & User cannot utilise the App
    & a. Graphics change brightness too rapidly \newline
      b. Graphics contain colours that are too bright \newline
    & a. App shall not change screen brightness unless the user chooses \newline
      b. Graphics shall be designed with dimmed or neutral colours
    & a. SFR1 \newline
      b. SFR2 \newline
    & H2 \\ \cline{2-7}
    % Windows Application continued
    & App does not receive data
    & Resistivity of sample cannot be calculated
    & a. Setup error \newline
      b. Hardware failure \newline
      c. Software connection error
    & a. Refer to H1 \newline
      b. Refer to H1 \newline
      c. Check that software is accessing the correct serial port
    & a. HWR1 \newline
      b. HWR2 \newline
      c. SWR1
    & H3  \\ \cline{2-7}
    %Windows Application continued
    & Calculated values are not correct
    & User receives inaccurate results
    & a. User altered measurements through interface \newline
      b. Software does not calculate values correctly
    & a. Prevent user from altering measurements received from hardware \newline
      b. Check that formulas for calculation used by software are correct
    & a. SCR1 \newline
      b. SWR2
    & H4  \\ \hline 

  \end{tabular}
\end{table}
\end{landscape}



\section{Safety and Security Requirements}

\wss{Newly discovered requirements.  These should also be added to the SRS.  (A
rationale design process how and why to fake it.)}

\section{Roadmap}

\wss{Which safety requirements will be implemented as part of the capstone timeline?
Which requirements will be implemented in the future?}

\end{document}